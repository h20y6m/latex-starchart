\documentclass[autodetect-engine,dvi=dvipdfmx,ja=standard,a4paper]{bxjsarticle}
\usepackage{amsmath}
\usepackage{lmodern}
\usepackage[colorlinks]{hyperref}
\usepackage{bxghost}

\usepackage{starchart}

\renewcommand\labelitemii{$\circ$}

\newcommand{\pkg}[1]{\eghostguarded{\textsf{#1}}}
\newcommand{\cmd}[1]{\eghostguarded{\texttt{\symbol{92}#1}}}
\newcommand{\Arg}[1]{\eghostguarded{\texttt{\symbol{123}}$\langle$\mbox{}#1\mbox{}$\rangle$}\texttt{\symbol{125}}}


\title{\pkg{starchart}パッケージ}
\author{h20y6m\thanks{\url{https://github.com/h20y6m}}}
\date{2020年3月20日}

\begin{document}

\maketitle

\section{概要}

\LaTeX で\pkg{tikz}を利用して星図を描画する。

\section{パッケージ読込}

\cmd{usepackage}命令を用いて読み込む。
\begin{quote}\begin{verbatim}
\usepackage[<オプション>]{starchart}
\end{verbatim}\end{quote}

オプションは以下のものがある。
\begin{itemize}
  \item \texttt{catalogue=<1/2/3/4/5/6>}\\
  読み込む星表のサイズ。
  \begin{itemize}
    \item \texttt{1}:一等星($\mathrm{Vmag}<1.5$)22に加えポラリス(北極星)と変光星ミラ含む。
    \item \texttt{2}:\texttt{1}に加え二等星($1.5 \le \mathrm{Vmag} < 2.5$)70を含む。
    \item \texttt{3}:\texttt{2}に加え三等星($2.5 \le \mathrm{Vmag} < 3.5$)192を含む。
    \item \texttt{4}:\texttt{3}に加え四等星($3.5 \le \mathrm{Vmag} < 4.5$)622を含む。
    \item \texttt{5}:\texttt{4}に加え五等星($4.5 \le \mathrm{Vmag} < 5.5$)1909を含む。
    \item \texttt{6}:\texttt{5}に加え六等星($5.5 \le \mathrm{Vmag} < 6.5$)5968を含む。
  \end{itemize} 
\end{itemize}

\section{機能}

\subsection{星表}

\begin{itemize}
  \item \cmd{clearstarcatalogue}\\
  星表からすべての星を削除する。
  \item \cmd{addstarcatalogue}\Arg{赤経}\Arg{赤緯}\Arg{視等級}\Arg{名前}\\
  星表に星を追加する。
  \begin{itemize}
    \item 赤経:赤経(時)を表す浮動小数式($0 \le \text{赤経} < 24$)。
    \item 赤緯:赤緯(度)を表す浮動小数式($-90 \le \text{赤緯} \le 90$)。
    \item 視等級:視等級を表す浮動小数式。
    \item 名前:名前を表すトークンリスト。
  \end{itemize}
\end{itemize}

\subsection{位置と時刻}

\begin{itemize}
  \item \cmd{setstartchartlocation}\Arg{経度}\Arg{緯度}\\
  観測地点の緯度・経度を設定する。
  \begin{itemize}
    \item 経度:東経を正、西経を負とする浮動小数式($-180 < \text{経度} \le 180$)。
    \item 緯度:北緯を正、南緯を負とする浮動小数式($-90 \le \text{緯度} \le 90$)。
  \end{itemize}
  \item \cmd{setstartcharttimezone}\Arg{タイムゾーンオフセット}\\
  タイムゾーンを設定する。
  \item \cmd{setstartchartdatetime}\Arg{年}\Arg{月}\Arg{日}\Arg{時}\Arg{分}\Arg{秒}\\
  観測時刻を設定する。
\end{itemize}

\subsection{描画}

\begin{itemize}
  \item \cmd{starchart}\Arg{サイズ}\\
  星図を描画する。
\end{itemize}

\section{サンプル}

東京の日本標準時2020年3月20日午後9時を描画する場合は以下のように記述する。
\begin{quote}\begin{verbatim}
\setstartcharttimezone{9}
\setstartchartdatetime{2020}{3}{20}{21}{0}{0}
\setstartchartlocation{139.69172}{35.68956}
\starchart{\textwidth}
\end{verbatim}\end{quote}
これをタイプセットすると図\ref{fig1}が得られる。

\begin{figure}
  \setstartcharttimezone{9}% JST
  \setstartchartdatetime{2020}{3}{20}{21}{0}{0}% 2020/3/20 21:00
  \setstartchartlocation{139.69172}{35.68956}% Tokyo
  \starchart{\textwidth}
  \caption{2020/3/20 21:00 JST\quad Tokyo}
  \label{fig1}
\end{figure}

\end{document}